\documentclass{article}
%\usepackage{ctex}
\usepackage{minted}
\usepackage{xcolor}
\usepackage{tocloft}
\usepackage{amsfonts}
\usepackage{enumitem}
\usepackage{multicol}
\usepackage[inner=0.5in,outer=0.6in,vmargin=1in,marginparwidth=1in]{geometry}
\usepackage{mdframed}
%------------------------------%
\usepackage{tikz-feynman}
\usepackage{fontspec}
\usepackage{tikz}
\usepackage{amsmath}
%\usepackage{hyperref}
\usepackage{microtype}   % Fine small typographical details

%% Make table of content heading smaller
\tikzfeynmanset{compat=1.0.0}
\newcommand{\institution}{ShanDong University}
\newcommand{\textline}{\noindent\makebox[\linewidth]{\rule{.8\paperwidth}{0.8pt}}}
\title{Lecture-1}


\definecolor{iceberg}{rgb}{0.44, 0.65, 0.82}
\definecolor{skyblue}{rgb}{0.53, 0.81, 0.92}
\definecolor{lightgray}{rgb}{0.83, 0.83, 0.83}
\newmdenv[linecolor=iceberg,innertopmargin=0.2cm,innerleftmargin=2pt,innerrightmargin=2pt,skipbelow=1cm,linewidth=2pt]{notes}

\begin{document}
\begin{titlepage}
    \begin{center}
    \makeatletter
          \vspace*{0.7em}
          \tikz\node[scale=1.5]{%
          \color{gray}\Huge\ttfamily \char`\{\textcolor{red!75!black}{\@title}\char`\}};
      
          \vspace{1.2em}
          {\large Version \texttt{0.7} \qquad \institution \qquad \today}
      
          \vspace{5em}
    \makeatother
    \begin{tabular}{l|r}{\footnote{from Tikz-feynman}}
            \feynmandiagram [large,vertical=a to b] {
              i1 [particle=\(\tilde W\)] -- [plain, boson] a -- [anti fermion] i2 [particle=\(q\)],
              a -- [charged scalar, edge label=\(\tilde q\)] b,
              f1 [particle=\(\tilde g\)] -- [plain, gluon] b -- [fermion] [particle=\(q\)],
              };
      
        \hspace{6.5em}
      
        &
      
        \hspace{6.5em}
        \feynmandiagram[large,vertical=a to b] {
      i1 [particle=\(e^{-}\)] -- [fermion] a -- [fermion] i2 [particle=\(e^{+}\)],
      a -- [photon, edge label=\(\gamma\), momentum'=\(k\)] b,
      f1 [particle=\(\mu^{+}\)] -- [fermion] b -- [fermion] f2 [particle=\(\mu^{-}\)],
      };
    \end{tabular}
    \end{center}
      
    \vspace{1.5em}
    \textline
    \begin{center}
    \textit{\Large What is more important is that you are willing to work hard;\\to devote the time needed;\\to ask questions when things don't make sense!}
    \end{center}
    \textline
\tableofcontents
\end{titlepage}
\section{Introduction}
\subsection{Purpose of Discussion}
\large{We will show how photons arise from the
quantization of the electromagnetic field and how massive, charged particles such as
electrons arise from the quantization of matter fields.There are electron fields,
but also quark fields, neutrino fields, gluon fields, W and Z-boson fields, Higgs fields
and a whole slew of others}
\subsection{Why We Need QFT}
\large{There is an unavoidable contraction between classical physics and modern physics(Local Interaction \& Action at a distance).To be more specific, 
the locality give us motivation to investigate two problems:}

\paragraph{Answer 1}
\large{The combination of quantum mechanics and special relativity
implies that particle number is not conserved}
\subsubsection{A simple \& interesting physical problem}
According to uncertainty principle, particles restricted inside a box satisfy:
\begin{align}\label{uncertainty}
    \Delta p \geq \hbar / L \\
    \Delta E \geq \hbar c / L
\end{align}
Following above equation, when the variation of energy reaches $\Delta E=2 m c^{2}$, there will be unavoidable new particles appear.
If we still obey the non-relativistic QM, the only explanation is the particle number is not conserved!

\paragraph{Answer 2}
\large{All particles of the same type are the same}
Is is unbelievable that the proton from the beginning of the universe and that detected on the 
earth are exactly the same.One explanation that might be offered is that there’s a sea
of proton "stuff" filling the universe and when we make a proton we somehow dip our
hand into this stuff and from it mould a proton
\subsection{What is QFT}
The rules for quantizing a field are no different from that transforming CM into QM,thus the basic degrees of freedom in
quantum field theory are operator valued functions of space and time. This means that
we are dealing with an infinite number of degrees of freedom

\section{Units}
What we are familiar is SI units, however natural units are generally applied in cosmology 
and high-energy physics.
\begin{equation}
    c=\hbar=\epsilon_{\circ}=k_{B}=1
\end{equation}
\begin{itemize}
    \item $c = 2.9979 \times 10^{8} \mathrm{m} / \mathrm{s}$ 
    \item $\hbar = 1.0546 \times 10^{-34} \mathrm{J} \mathrm{s}$
    \item $\epsilon_{\mathrm{o}} = 8.8542 \times 10^{-12} \mathrm{A}^{2} \mathrm{s}^{4} \mathrm{kg}^{-1} \mathrm{m}^{-3}$
    \item $k_{B}=\text { Boltzmann constant }=1.3806 \times 10^{-23} \mathrm{J} \mathrm{K}^{-1}$
\end{itemize}
A simple formula to convert numerical value between different units:
\begin{equation}\footnote{E can arbitrarily chosen energy unit, popular as GeV}
    \left(\mathrm{kg}^{\alpha} \mathrm{m}^{\beta} \mathrm{s}^{\gamma}\right) \rightarrow (E)^{\alpha-\beta-\gamma} \hbar^{\beta+\gamma} c^{\beta-2 \alpha}
\end{equation}
\subsubsection{Plank Mass}
Let us examine the Einstein Equations with cosmological constant
\begin{equation*}
    R_{\mu \nu}-\frac{1}{2} R g_{\mu \nu}+\Lambda g_{\mu \nu}=\frac{8 \pi G}{c^4}T_{\mu \nu}
\end{equation*}
In natural units, the above equation becomes:
\begin{equation}
    R_{\mu \nu}-\frac{1}{2} R g_{\mu \nu}+\Lambda g_{\mu \nu}={8 \pi G}T_{\mu \nu}
\end{equation}
If we define the planck mass as $m_p = \sqrt{\frac{\hbar\times c}{G}}=2.7164\times 10^{-8}kg$,
then we can calculate $m_{p}=1.2209 \times 10^{19} \mathrm{GeV}$ \textbf{Calculate It!}

\section{Classical Field Review}
\paragraph{Concepts}
\large{When constructing the Lagrangian for a field distribution(discussing only scalar \& non-interacting field),
the spacetime coordinates are just index, comparing to the individual particles.}
The dynamical variable can be treated as $\varphi ( \mathbf { r } , t )$, given a matter distribution:
$\rho = \rho ( \mathbf { r } , t )$, 
or we can also treat the field variables as a map:$\varphi : \mathbf { R } ^ { 4 } \rightarrow \mathbf { R }$, in a typical spacetime domain:
% insert a manifold diagram 
% todo recommend using Tikz
\begin{equation}\label{domain}
    \mathcal { R } = \{ ( t , x , y , z ) | t _ { 1 } \leq t \leq t _ { 2 } , - \infty < x , y , z < \infty \}
\end{equation}
Thus we are dealing with a system with an infinite number of 
degrees of freedom, compared to CM. For convenience, we can treat the field quantity
defined on spacetime as a function on $4-D$ manifold.

\subsection{Electromagnetic Field Description}
From what we are really used to write "Maxwell Equations":
\begin{align}
    \vec{\nabla} \cdot \vec{E} &=4 \pi \rho \\
    \vec{\nabla}\times\vec{B}-\frac{1}{c} \frac{\partial\vec{E}}{\partial t} &=\frac{4\pi}{c}\vec{J}\\
    \vec{\nabla}\cdot \vec{B} &=0\\
    \vec{\nabla}\times\vec{E}+\frac{1}{c}\frac{\partial\vec{B}}{\partial t} &=0
\end{align}
The above equation based on the Gaussian Units:
\begin{center}
\begin{itemize}
    \item $\vec{E}\longrightarrow\frac{1}{\sqrt{4\pi\epsilon_{0}}}\vec{E}$, $\vec{B}\longrightarrow\sqrt{\frac{\mu_{0}}{4 \pi}}\vec{B}$ $\rho\longrightarrow\sqrt{4\pi\epsilon_{0}}\rho$ , $\vec{J}\longrightarrow\sqrt{4\pi\epsilon_{0}}\vec{J}$
\end{itemize}
\end{center}
As a result, we can write the Maxwell Equations as:
\begin{equation}\footnote{$A^{\mu}(\vec{x}, t)=(\phi, \vec{A})$}
    \vec{E}=-\vec{\nabla} \phi-\frac{\partial\vec{A}}{\partial t}, \quad\vec{B}=\vec{\nabla}\times\vec{A}
\end{equation}
\subsection{Lagrangian}
Similar in the CM, the Lagrangian \& Hamiltonian are important to catch the system
dynamics. In Field theory, the Lagrangian can more or less treated as the hypersurface
integral:
\begin{equation}
    L(t)=\int d^{3} x \mathcal{L}\left(\phi_{a},\partial_{\mu}\phi_{a}\right)
\end{equation}
The action is just:
\begin{equation}
    S=\int_{t_{1}}^{t_{2}} d t \int d^{3} x \mathcal{L}=\int d^{4} x \mathcal{L}
\end{equation}
We notice that the constructed Lagrangian is only related to first derivative with
respect to spacetime coordinates. But in practice, it is more common to find relationship
between higher derivatives
\subsection{Variational Principle}
In physics, fundamental theories always arise from a variational principle.Ultimately this stems from their roots as quantum mechanical systems, 
as can be seen from Feynman’s path integral formalism.\\
In a typical spacetime domain, via the Lagrangian we construct the action:$S = S [ \varphi ]$ which is a functional of field $\varphi$.
Like the procedure in particle circumstances,the variational principle goes as follows:\\
Consider any family of functions, labeled by a parameter $\lambda$, which includes some given function $\lambda_0$ at
$\lambda = 0$,we can think of $\varphi_\lambda$ as defining a curve in the manifold which
passes through the point $\varphi_0$,getting the action:
\begin{equation}
    S ( \lambda ) : = S \left[ \varphi _ { \lambda } \right]
\end{equation}
the critical point $\lambda = 0$ is just the same as in real number space, (just manifold space),
and the turning points of the functional $S [ \varphi ]$ correspond
to functions on spacetime which solve the Euler-Lagrangian equations.\\
Let us consider a curve that passes through a putative critical point $\varphi$ (say $\lambda = 0$) then:
\begin{equation}
    \varphi _ { \lambda } :  \lambda = 0,\varphi_0=\varphi.
\end{equation}
If $\varphi _ { \lambda = 0 } \equiv \varphi$ is a critical point in the "field manifold",then:
\begin{equation}
    \delta S : = \left( \frac { d S ( \lambda ) } { d \lambda } \right) _ { \lambda = 0 } = 0
\end{equation}
\begin{equation}
    \delta \varphi : = \left( \frac { d \varphi ( \lambda ) } { d \lambda } \right) _ { \lambda = 0 }
\end{equation}
\begin{itemize}
    \item $\delta S$ the first variation of the action.
    \item $\delta \varphi$ the variation of $\varphi$.
\end{itemize}
In the general case, we can reduce the variation of the action based on the Lagrangian density:
\begin{equation}
    \delta S [ \varphi ] = \int _ { \mathcal { R } } d ^ { 4 } x F ( x ) \delta \varphi ( x ), F ( x ) \equiv \frac { \delta S } { \delta \varphi ( x ) }
\end{equation}
The general variational principle tells us that $\frac { \delta S } { \delta \varphi } = 0$ can be obtained.
\subsection{Euler-Lagrangian Equations}
Generally, what we discuss is the local field theory(global theory is deep and different)
\subsubsection{Example}
Take the KG field as an example,the time-dependent total Lagrangian on a hypersurface and total Lagrangian for KG field is:
\begin{align}
    L &= \int _ { \mathbf { R } ^ { 3 } } d ^ { 3 } x \frac { 1 } { 2 } \left( \varphi _ { , t } ^ { 2 } - ( \nabla \varphi ) ^ { 2 } - m ^ { 2 } \varphi ^ { 2 } \right)\\
    S [ \varphi ] &= \int _ { t _ { 1 } } ^ { t _ { 2 } } d t L
\end{align}
The variation of Lagrangian is:
\begin{equation}\label{boundary}
\begin{aligned} 
\delta \mathcal { L } & = \varphi _ { , t } \delta \varphi _ { , t } - \nabla \varphi \cdot \nabla \delta \varphi - m ^ { 2 } \varphi \delta \varphi \\ 
& = \left( - \varphi _ { , t t } + \nabla ^ { 2 } \varphi - m ^ { 2 } \varphi \right) \delta \varphi + \frac { \partial } { \partial x ^ { \alpha } } V ^ { \alpha }
\end{aligned}
\end{equation}
where: $V ^ { 0 } = \varphi _ { , t } \delta \varphi , \quad V ^ { i } = - ( \nabla \varphi ) ^ { i } \delta \varphi$.When taking the boundary conditions into account, 
the last term in (\ref{boundary}) usually vanish in the process of integral. 
\subsubsection{General Case}
For any formula like $F \left( x , \varphi , \varphi _ { \alpha } \right)$, we give the definition of total derivative with respect to some coordinates:
\begin{equation}
    D _ { \alpha } F \left( x , \varphi , \varphi _ { \alpha } \right) = \frac { \partial F } { \partial x ^ { \alpha } } + \frac { \partial F } { \partial \varphi } \varphi _ { \alpha } + \frac { \partial F } { \partial \varphi _ { \beta } } \varphi _ { \alpha \beta }
\end{equation}
Following the procedure, we define the variation of the Lagrangian density as:
\begin{equation}
    \delta \mathcal { L } : = \frac { \partial \mathcal { L } } { \partial \varphi } \delta \varphi + \frac { \partial \mathcal { L } } { \partial \varphi _ { \alpha } } \delta \varphi _ { \alpha }
\end{equation}
then, we can rewrite the expression as:
\begin{equation}
    \delta \mathcal { L } = \left( \frac { \partial \mathcal { L } } { \partial \varphi } - D _ { \alpha } \frac { \partial \mathcal { L } } { \partial \varphi _ { \alpha } } \right) \delta \varphi + D _ { \alpha } V ^ { \alpha }, V ^ { \alpha } = \frac { \partial \mathcal { L } } { \partial \varphi _ { \alpha } } \delta \varphi
\end{equation}

\end{document}